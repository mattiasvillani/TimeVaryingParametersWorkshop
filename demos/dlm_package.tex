% Options for packages loaded elsewhere
% Options for packages loaded elsewhere
\PassOptionsToPackage{unicode}{hyperref}
\PassOptionsToPackage{hyphens}{url}
\PassOptionsToPackage{dvipsnames,svgnames,x11names}{xcolor}
%
\documentclass[
  letterpaper,
  DIV=11,
  numbers=noendperiod]{scrartcl}
\usepackage{xcolor}
\usepackage{amsmath,amssymb}
\setcounter{secnumdepth}{-\maxdimen} % remove section numbering
\usepackage{iftex}
\ifPDFTeX
  \usepackage[T1]{fontenc}
  \usepackage[utf8]{inputenc}
  \usepackage{textcomp} % provide euro and other symbols
\else % if luatex or xetex
  \usepackage{unicode-math} % this also loads fontspec
  \defaultfontfeatures{Scale=MatchLowercase}
  \defaultfontfeatures[\rmfamily]{Ligatures=TeX,Scale=1}
\fi
\usepackage{lmodern}
\ifPDFTeX\else
  % xetex/luatex font selection
\fi
% Use upquote if available, for straight quotes in verbatim environments
\IfFileExists{upquote.sty}{\usepackage{upquote}}{}
\IfFileExists{microtype.sty}{% use microtype if available
  \usepackage[]{microtype}
  \UseMicrotypeSet[protrusion]{basicmath} % disable protrusion for tt fonts
}{}
\makeatletter
\@ifundefined{KOMAClassName}{% if non-KOMA class
  \IfFileExists{parskip.sty}{%
    \usepackage{parskip}
  }{% else
    \setlength{\parindent}{0pt}
    \setlength{\parskip}{6pt plus 2pt minus 1pt}}
}{% if KOMA class
  \KOMAoptions{parskip=half}}
\makeatother
% Make \paragraph and \subparagraph free-standing
\makeatletter
\ifx\paragraph\undefined\else
  \let\oldparagraph\paragraph
  \renewcommand{\paragraph}{
    \@ifstar
      \xxxParagraphStar
      \xxxParagraphNoStar
  }
  \newcommand{\xxxParagraphStar}[1]{\oldparagraph*{#1}\mbox{}}
  \newcommand{\xxxParagraphNoStar}[1]{\oldparagraph{#1}\mbox{}}
\fi
\ifx\subparagraph\undefined\else
  \let\oldsubparagraph\subparagraph
  \renewcommand{\subparagraph}{
    \@ifstar
      \xxxSubParagraphStar
      \xxxSubParagraphNoStar
  }
  \newcommand{\xxxSubParagraphStar}[1]{\oldsubparagraph*{#1}\mbox{}}
  \newcommand{\xxxSubParagraphNoStar}[1]{\oldsubparagraph{#1}\mbox{}}
\fi
\makeatother

\usepackage{color}
\usepackage{fancyvrb}
\newcommand{\VerbBar}{|}
\newcommand{\VERB}{\Verb[commandchars=\\\{\}]}
\DefineVerbatimEnvironment{Highlighting}{Verbatim}{commandchars=\\\{\}}
% Add ',fontsize=\small' for more characters per line
\usepackage{framed}
\definecolor{shadecolor}{RGB}{241,243,245}
\newenvironment{Shaded}{\begin{snugshade}}{\end{snugshade}}
\newcommand{\AlertTok}[1]{\textcolor[rgb]{0.68,0.00,0.00}{#1}}
\newcommand{\AnnotationTok}[1]{\textcolor[rgb]{0.37,0.37,0.37}{#1}}
\newcommand{\AttributeTok}[1]{\textcolor[rgb]{0.40,0.45,0.13}{#1}}
\newcommand{\BaseNTok}[1]{\textcolor[rgb]{0.68,0.00,0.00}{#1}}
\newcommand{\BuiltInTok}[1]{\textcolor[rgb]{0.00,0.23,0.31}{#1}}
\newcommand{\CharTok}[1]{\textcolor[rgb]{0.13,0.47,0.30}{#1}}
\newcommand{\CommentTok}[1]{\textcolor[rgb]{0.37,0.37,0.37}{#1}}
\newcommand{\CommentVarTok}[1]{\textcolor[rgb]{0.37,0.37,0.37}{\textit{#1}}}
\newcommand{\ConstantTok}[1]{\textcolor[rgb]{0.56,0.35,0.01}{#1}}
\newcommand{\ControlFlowTok}[1]{\textcolor[rgb]{0.00,0.23,0.31}{\textbf{#1}}}
\newcommand{\DataTypeTok}[1]{\textcolor[rgb]{0.68,0.00,0.00}{#1}}
\newcommand{\DecValTok}[1]{\textcolor[rgb]{0.68,0.00,0.00}{#1}}
\newcommand{\DocumentationTok}[1]{\textcolor[rgb]{0.37,0.37,0.37}{\textit{#1}}}
\newcommand{\ErrorTok}[1]{\textcolor[rgb]{0.68,0.00,0.00}{#1}}
\newcommand{\ExtensionTok}[1]{\textcolor[rgb]{0.00,0.23,0.31}{#1}}
\newcommand{\FloatTok}[1]{\textcolor[rgb]{0.68,0.00,0.00}{#1}}
\newcommand{\FunctionTok}[1]{\textcolor[rgb]{0.28,0.35,0.67}{#1}}
\newcommand{\ImportTok}[1]{\textcolor[rgb]{0.00,0.46,0.62}{#1}}
\newcommand{\InformationTok}[1]{\textcolor[rgb]{0.37,0.37,0.37}{#1}}
\newcommand{\KeywordTok}[1]{\textcolor[rgb]{0.00,0.23,0.31}{\textbf{#1}}}
\newcommand{\NormalTok}[1]{\textcolor[rgb]{0.00,0.23,0.31}{#1}}
\newcommand{\OperatorTok}[1]{\textcolor[rgb]{0.37,0.37,0.37}{#1}}
\newcommand{\OtherTok}[1]{\textcolor[rgb]{0.00,0.23,0.31}{#1}}
\newcommand{\PreprocessorTok}[1]{\textcolor[rgb]{0.68,0.00,0.00}{#1}}
\newcommand{\RegionMarkerTok}[1]{\textcolor[rgb]{0.00,0.23,0.31}{#1}}
\newcommand{\SpecialCharTok}[1]{\textcolor[rgb]{0.37,0.37,0.37}{#1}}
\newcommand{\SpecialStringTok}[1]{\textcolor[rgb]{0.13,0.47,0.30}{#1}}
\newcommand{\StringTok}[1]{\textcolor[rgb]{0.13,0.47,0.30}{#1}}
\newcommand{\VariableTok}[1]{\textcolor[rgb]{0.07,0.07,0.07}{#1}}
\newcommand{\VerbatimStringTok}[1]{\textcolor[rgb]{0.13,0.47,0.30}{#1}}
\newcommand{\WarningTok}[1]{\textcolor[rgb]{0.37,0.37,0.37}{\textit{#1}}}

\usepackage{longtable,booktabs,array}
\usepackage{calc} % for calculating minipage widths
% Correct order of tables after \paragraph or \subparagraph
\usepackage{etoolbox}
\makeatletter
\patchcmd\longtable{\par}{\if@noskipsec\mbox{}\fi\par}{}{}
\makeatother
% Allow footnotes in longtable head/foot
\IfFileExists{footnotehyper.sty}{\usepackage{footnotehyper}}{\usepackage{footnote}}
\makesavenoteenv{longtable}
\usepackage{graphicx}
\makeatletter
\newsavebox\pandoc@box
\newcommand*\pandocbounded[1]{% scales image to fit in text height/width
  \sbox\pandoc@box{#1}%
  \Gscale@div\@tempa{\textheight}{\dimexpr\ht\pandoc@box+\dp\pandoc@box\relax}%
  \Gscale@div\@tempb{\linewidth}{\wd\pandoc@box}%
  \ifdim\@tempb\p@<\@tempa\p@\let\@tempa\@tempb\fi% select the smaller of both
  \ifdim\@tempa\p@<\p@\scalebox{\@tempa}{\usebox\pandoc@box}%
  \else\usebox{\pandoc@box}%
  \fi%
}
% Set default figure placement to htbp
\def\fps@figure{htbp}
\makeatother





\setlength{\emergencystretch}{3em} % prevent overfull lines

\providecommand{\tightlist}{%
  \setlength{\itemsep}{0pt}\setlength{\parskip}{0pt}}



 


\KOMAoption{captions}{tableheading}
\makeatletter
\@ifpackageloaded{caption}{}{\usepackage{caption}}
\AtBeginDocument{%
\ifdefined\contentsname
  \renewcommand*\contentsname{Table of contents}
\else
  \newcommand\contentsname{Table of contents}
\fi
\ifdefined\listfigurename
  \renewcommand*\listfigurename{List of Figures}
\else
  \newcommand\listfigurename{List of Figures}
\fi
\ifdefined\listtablename
  \renewcommand*\listtablename{List of Tables}
\else
  \newcommand\listtablename{List of Tables}
\fi
\ifdefined\figurename
  \renewcommand*\figurename{Figure}
\else
  \newcommand\figurename{Figure}
\fi
\ifdefined\tablename
  \renewcommand*\tablename{Table}
\else
  \newcommand\tablename{Table}
\fi
}
\@ifpackageloaded{float}{}{\usepackage{float}}
\floatstyle{ruled}
\@ifundefined{c@chapter}{\newfloat{codelisting}{h}{lop}}{\newfloat{codelisting}{h}{lop}[chapter]}
\floatname{codelisting}{Listing}
\newcommand*\listoflistings{\listof{codelisting}{List of Listings}}
\makeatother
\makeatletter
\makeatother
\makeatletter
\@ifpackageloaded{caption}{}{\usepackage{caption}}
\@ifpackageloaded{subcaption}{}{\usepackage{subcaption}}
\makeatother
\usepackage{bookmark}
\IfFileExists{xurl.sty}{\usepackage{xurl}}{} % add URL line breaks if available
\urlstyle{same}
\hypersetup{
  pdftitle={Demo - the dlm package for state-space models in R},
  pdfauthor={Mattias Villani},
  colorlinks=true,
  linkcolor={blue},
  filecolor={Maroon},
  citecolor={Blue},
  urlcolor={Blue},
  pdfcreator={LaTeX via pandoc}}


\title{Demo - the dlm package for state-space models in R}
\author{Mattias Villani}
\date{}
\begin{document}
\maketitle

\renewcommand*\contentsname{Table of contents}
{
\hypersetup{linkcolor=}
\setcounter{tocdepth}{3}
\tableofcontents
}

\begin{quote}
This notebook illustrates the use of the dlm package in R. How to set up
a state-space model, and how to do Kalman filtering, smoothing and
forecasting. The local level model for the well-known Nile river data is
used as the running example. The final bonus section shows how the
Kalman filter is implemented from scratch in R.
\end{quote}

\subsection{Local level model}\label{local-level-model}

The \textbf{local level} model has a constantly changing mean following
a random walk model:

\[y_t = \mu_t + \varepsilon_t,\qquad \varepsilon_t \sim N(0,\sigma_\varepsilon^2)\]

\[\mu_t = \mu_{t-1} + \eta_t,\qquad \eta_t \sim N(0,\sigma_\eta^2)\]

which models the observed time series \(y_t\) as a mean \(\mu_t\) plus a
random \textbf{measurement error} or disturbance \(\varepsilon_t\). The
mean \(\mu_t\) evolves over time as a random walk driven by
\textbf{innovations} \(\eta_t\).

\subsection{\texorpdfstring{The \texttt{dlm} package in
R}{The dlm package in R}}\label{the-dlm-package-in-r}

The dlm package uses the following notation for a \textbf{state-space
model} for a univariate time series \(y_t\) with a \emph{state} vector
\(\boldsymbol{\theta}_t\):

\[
\begin{align}
y_t &= \boldsymbol{F} \boldsymbol{\theta}_t + v_t,\hspace{1.5cm} v_t \sim N(\boldsymbol{0},\boldsymbol{v})  \\
\boldsymbol{\theta}_t &= \boldsymbol{G} \boldsymbol{\theta}_{t-1} + \boldsymbol{w}_t, \qquad \boldsymbol{w}_t \sim N(\boldsymbol{0},\boldsymbol{W})
\end{align}
\]

For example, the local level model is a state-space model with a single
scalar state variable \(\boldsymbol{\theta}_t = \mu_t\) and parameters

\[
\begin{align}
 \boldsymbol{F} &= 1 \\
 \boldsymbol{G} &= 1  \\
 \boldsymbol{V} &= \sigma_\varepsilon^2 \\
\boldsymbol{W} &= \sigma_\eta^2
\end{align}
\]

The dlm package is a user-friendly R package for analyzing some
state-space models. The package has a nice
\href{https://cran.r-project.org/web/packages/dlm/vignettes/dlm.pdf}{vignette}
that is worth reading if you plan to use the package more seriously.

\paragraph{Filtering}\label{filtering}

Let's first do some filtering in the \texttt{dlm} package. Start by
loading the \texttt{dlm} package:

\begin{Shaded}
\begin{Highlighting}[]
\CommentTok{\#install.packages("dlm") \# uncomment the first time to install.}
\FunctionTok{library}\NormalTok{(dlm)}
\end{Highlighting}
\end{Shaded}

We now need to tell the \texttt{dlm} package what kind of state-space
model we want to estimate. The means setting up the matrices
\(\boldsymbol{F}\), \(\boldsymbol{G}\), \(\boldsymbol{V}\) and
\(\boldsymbol{W}\) for the local level model. We will first set the two
static parameters to: \(\sigma_\varepsilon^2 = 100^2\) and
\(\sigma_\eta^2 = 100^2\). Later we estimate these parameters by maximum
likelihood. Here is how you setup the local level model in the
\texttt{dlm} package:

\begin{Shaded}
\begin{Highlighting}[]
\NormalTok{model }\OtherTok{=} \FunctionTok{dlm}\NormalTok{(}\AttributeTok{FF =} \DecValTok{1}\NormalTok{, }\AttributeTok{V =} \DecValTok{100}\SpecialCharTok{\^{}}\DecValTok{2}\NormalTok{, }\AttributeTok{GG =} \DecValTok{1}\NormalTok{, }\AttributeTok{W =} \DecValTok{100}\SpecialCharTok{\^{}}\DecValTok{2}\NormalTok{, }\AttributeTok{m0 =} \DecValTok{1000}\NormalTok{, }\AttributeTok{C0 =} \DecValTok{1000}\SpecialCharTok{\^{}}\DecValTok{2}\NormalTok{)}
\end{Highlighting}
\end{Shaded}

The two last arguments to the dlm function is the prior mean
(\texttt{m0}) and (co)variance (\texttt{C0}) for the state at time
\(t=0\).

Compute the filtering estimate using the Kalman filter and plot the
result

\begin{Shaded}
\begin{Highlighting}[]
\NormalTok{nileFilter }\OtherTok{\textless{}{-}} \FunctionTok{dlmFilter}\NormalTok{(Nile, model)}
\FunctionTok{plot}\NormalTok{(Nile, }\AttributeTok{type =} \StringTok{\textquotesingle{}l\textquotesingle{}}\NormalTok{, }\AttributeTok{col =} \StringTok{"steelblue"}\NormalTok{)}
\FunctionTok{lines}\NormalTok{(}\FunctionTok{dropFirst}\NormalTok{(nileFilter}\SpecialCharTok{$}\NormalTok{m), }\AttributeTok{type =} \StringTok{\textquotesingle{}l\textquotesingle{}}\NormalTok{, }\AttributeTok{col =} \StringTok{"orange"}\NormalTok{)}
\FunctionTok{legend}\NormalTok{(}\StringTok{"bottomleft"}\NormalTok{, }\AttributeTok{legend =} \FunctionTok{c}\NormalTok{(}\StringTok{"Observed"}\NormalTok{, }\StringTok{"Filtered"}\NormalTok{), }\AttributeTok{lty =} \DecValTok{1}\NormalTok{, }
    \AttributeTok{col =} \FunctionTok{c}\NormalTok{(}\StringTok{"steelblue"}\NormalTok{, }\StringTok{"orange"}\NormalTok{))}
\end{Highlighting}
\end{Shaded}

\pandocbounded{\includegraphics[keepaspectratio]{dlm_package_files/figure-pdf/unnamed-chunk-3-1.pdf}}

The dlm package also infers the initial value of the state at time
\(t=0\). By using the \texttt{dropFirst} command we only plot the
filtering posterior for \(t=1,\ldots,T\).

\subsubsection{Parameter estimation by maximum
likelihood}\label{parameter-estimation-by-maximum-likelihood}

The parameters \(\sigma_\varepsilon^2\) and \(\sigma_\eta^2\) were just
set to some values above. The function \texttt{dlmMLE} estimates these
parameters by maximum likelihood, but we need to set up a model build
object so the \texttt{dlm} package knows which parameter to estimate. We
reparameterize the two variances using the exponential function to
ensure that the estimated variances are positive.

\begin{Shaded}
\begin{Highlighting}[]
\NormalTok{ modelBuild }\OtherTok{\textless{}{-}} \ControlFlowTok{function}\NormalTok{(param) \{}
   \FunctionTok{dlm}\NormalTok{(}\AttributeTok{FF =} \DecValTok{1}\NormalTok{, }\AttributeTok{V =} \FunctionTok{exp}\NormalTok{(param[}\DecValTok{1}\NormalTok{]), }\AttributeTok{GG =} \DecValTok{1}\NormalTok{, }\AttributeTok{W =} \FunctionTok{exp}\NormalTok{(param[}\DecValTok{2}\NormalTok{]), }\AttributeTok{m0 =} \DecValTok{1000}\NormalTok{, }\AttributeTok{C0 =} \DecValTok{1000}\SpecialCharTok{\^{}}\DecValTok{2}\NormalTok{)}
\NormalTok{ \}}
\NormalTok{ fit }\OtherTok{\textless{}{-}} \FunctionTok{dlmMLE}\NormalTok{(Nile, }\AttributeTok{parm =} \FunctionTok{c}\NormalTok{(}\DecValTok{0}\NormalTok{,}\DecValTok{0}\NormalTok{), }\AttributeTok{build =}\NormalTok{ modelBuild)}
\end{Highlighting}
\end{Shaded}

where \texttt{parm} is a vector with initial values for the two
parameters (on the log scale, since we use exponential functions to
ensure positive variances).

We need to take the exponential of the MLEs to get the estimated
variance parameters.

\begin{Shaded}
\begin{Highlighting}[]
 \FunctionTok{exp}\NormalTok{(fit}\SpecialCharTok{$}\NormalTok{par)}
\end{Highlighting}
\end{Shaded}

\begin{verbatim}
[1] 15101.339  1467.049
\end{verbatim}

or the square roots, to get the maximum likelihood estimates of the
standard deviations

\begin{Shaded}
\begin{Highlighting}[]
\FunctionTok{sqrt}\NormalTok{(}\FunctionTok{exp}\NormalTok{(fit}\SpecialCharTok{$}\NormalTok{par))}
\end{Highlighting}
\end{Shaded}

\begin{verbatim}
[1] 122.88750  38.30208
\end{verbatim}

We can redo the filtering, this time using the maximum likelihood
estimates of the parameters:

\begin{Shaded}
\begin{Highlighting}[]
\NormalTok{model\_mle }\OtherTok{=} \FunctionTok{dlm}\NormalTok{(}\AttributeTok{FF =} \DecValTok{1}\NormalTok{, }\AttributeTok{V =} \FunctionTok{exp}\NormalTok{(fit}\SpecialCharTok{$}\NormalTok{par[}\DecValTok{1}\NormalTok{]), }\AttributeTok{GG =} \DecValTok{1}\NormalTok{, }\AttributeTok{W =} \FunctionTok{exp}\NormalTok{(fit}\SpecialCharTok{$}\NormalTok{par[}\DecValTok{2}\NormalTok{]), }\AttributeTok{m0 =} \DecValTok{1000}\NormalTok{, }\AttributeTok{C0 =} \DecValTok{1000}\SpecialCharTok{\^{}}\DecValTok{2}\NormalTok{)}
\NormalTok{nileFilter }\OtherTok{\textless{}{-}} \FunctionTok{dlmFilter}\NormalTok{(Nile, model\_mle)}
\FunctionTok{plot}\NormalTok{(Nile, }\AttributeTok{type =} \StringTok{\textquotesingle{}l\textquotesingle{}}\NormalTok{, }\AttributeTok{col =} \StringTok{"steelblue"}\NormalTok{, }\AttributeTok{lwd =} \FloatTok{1.5}\NormalTok{)}
\FunctionTok{lines}\NormalTok{(}\FunctionTok{dropFirst}\NormalTok{(nileFilter}\SpecialCharTok{$}\NormalTok{m), }\AttributeTok{type =} \StringTok{\textquotesingle{}l\textquotesingle{}}\NormalTok{, }\AttributeTok{col =} \StringTok{"orange"}\NormalTok{, }\AttributeTok{lwd =} \FloatTok{1.5}\NormalTok{)}
\FunctionTok{legend}\NormalTok{(}\StringTok{"bottomleft"}\NormalTok{, }\AttributeTok{legend =} \FunctionTok{c}\NormalTok{(}\StringTok{"Observed"}\NormalTok{, }\StringTok{"Filtered"}\NormalTok{), }\AttributeTok{lwd =} \FloatTok{1.5}\NormalTok{, }\AttributeTok{lty =} \DecValTok{1}\NormalTok{, }
    \AttributeTok{col =} \FunctionTok{c}\NormalTok{(}\StringTok{"steelblue"}\NormalTok{, }\StringTok{"orange"}\NormalTok{))}
\end{Highlighting}
\end{Shaded}

\pandocbounded{\includegraphics[keepaspectratio]{dlm_package_files/figure-pdf/unnamed-chunk-7-1.pdf}}

\subsubsection{Smoothing}\label{smoothing}

We can also use the \texttt{dlm} package to compute the smoothed
retrospective estimates of the local level \(\mu_t\) at time \(t\) using
all the data from \(t=1\) until the end of the time series \(T\). Here
is the smoothing results for the Nile data, using the function
\texttt{dlmSmooth} from the \texttt{dlm} package. The filtered estimates
are also shown.

\begin{Shaded}
\begin{Highlighting}[]
\NormalTok{nileSmooth }\OtherTok{\textless{}{-}} \FunctionTok{dlmSmooth}\NormalTok{(Nile, model\_mle)}
\FunctionTok{plot}\NormalTok{(Nile, }\AttributeTok{type =} \StringTok{\textquotesingle{}l\textquotesingle{}}\NormalTok{, }\AttributeTok{col =} \StringTok{"steelblue"}\NormalTok{, }\AttributeTok{lwd =} \FloatTok{1.5}\NormalTok{)}
\FunctionTok{lines}\NormalTok{(}\FunctionTok{dropFirst}\NormalTok{(nileFilter}\SpecialCharTok{$}\NormalTok{m), }\AttributeTok{type =} \StringTok{\textquotesingle{}l\textquotesingle{}}\NormalTok{, }\AttributeTok{col =} \StringTok{"orange"}\NormalTok{, }\AttributeTok{lwd =} \FloatTok{1.5}\NormalTok{)}
\FunctionTok{lines}\NormalTok{(}\FunctionTok{dropFirst}\NormalTok{(nileSmooth}\SpecialCharTok{$}\NormalTok{s), }\AttributeTok{type =} \StringTok{\textquotesingle{}l\textquotesingle{}}\NormalTok{, }\AttributeTok{col =} \StringTok{"red"}\NormalTok{, }\AttributeTok{lwd =} \FloatTok{1.5}\NormalTok{)}
\FunctionTok{legend}\NormalTok{(}\StringTok{"bottomleft"}\NormalTok{, }\AttributeTok{legend =} \FunctionTok{c}\NormalTok{(}\StringTok{"Observed"}\NormalTok{, }\StringTok{"Filtered"}\NormalTok{,}\StringTok{"Smoothed"}\NormalTok{), }\AttributeTok{lty =} \DecValTok{1}\NormalTok{, }\AttributeTok{lwd =} \FloatTok{1.5}\NormalTok{, }\AttributeTok{col =} \FunctionTok{c}\NormalTok{(}\StringTok{"steelblue"}\NormalTok{, }\StringTok{"orange"}\NormalTok{, }\StringTok{"red"}\NormalTok{))}
\end{Highlighting}
\end{Shaded}

\pandocbounded{\includegraphics[keepaspectratio]{dlm_package_files/figure-pdf/unnamed-chunk-8-1.pdf}}

\subsubsection{Forecasting}\label{forecasting}

We can also use state-space models for forecasting. Here is how it is
done in the \texttt{dlm} package.

\begin{Shaded}
\begin{Highlighting}[]
\NormalTok{nileFore }\OtherTok{\textless{}{-}} \FunctionTok{dlmForecast}\NormalTok{(nileFilter, }\AttributeTok{nAhead =} \DecValTok{5}\NormalTok{)}
\NormalTok{sqrtR }\OtherTok{\textless{}{-}} \FunctionTok{sapply}\NormalTok{(nileFore}\SpecialCharTok{$}\NormalTok{R, }\ControlFlowTok{function}\NormalTok{(x) }\FunctionTok{sqrt}\NormalTok{(x))}
\NormalTok{pl }\OtherTok{\textless{}{-}}\NormalTok{ nileFore}\SpecialCharTok{$}\NormalTok{a[,}\DecValTok{1}\NormalTok{] }\SpecialCharTok{+} \FunctionTok{qnorm}\NormalTok{(}\FloatTok{0.05}\NormalTok{, }\AttributeTok{sd =}\NormalTok{ sqrtR)}
\NormalTok{pu }\OtherTok{\textless{}{-}}\NormalTok{ nileFore}\SpecialCharTok{$}\NormalTok{a[,}\DecValTok{1}\NormalTok{] }\SpecialCharTok{+} \FunctionTok{qnorm}\NormalTok{(}\FloatTok{0.95}\NormalTok{, }\AttributeTok{sd =}\NormalTok{ sqrtR)}
\NormalTok{x }\OtherTok{\textless{}{-}} \FunctionTok{ts.union}\NormalTok{(}\FunctionTok{window}\NormalTok{(Nile, }\AttributeTok{start =} \FunctionTok{c}\NormalTok{(}\DecValTok{1900}\NormalTok{, }\DecValTok{1}\NormalTok{)),}
              \FunctionTok{window}\NormalTok{(nileSmooth}\SpecialCharTok{$}\NormalTok{s, }\AttributeTok{start =} \FunctionTok{c}\NormalTok{(}\DecValTok{1900}\NormalTok{, }\DecValTok{1}\NormalTok{)), }
\NormalTok{              nileFore}\SpecialCharTok{$}\NormalTok{a, pl, pu)}

\FunctionTok{plot}\NormalTok{(x, }\AttributeTok{plot.type =} \StringTok{"single"}\NormalTok{, }\AttributeTok{type =} \StringTok{\textquotesingle{}o\textquotesingle{}}\NormalTok{, }\AttributeTok{pch =} \FunctionTok{c}\NormalTok{(}\ConstantTok{NA}\NormalTok{, }\ConstantTok{NA}\NormalTok{, }\ConstantTok{NA}\NormalTok{, }\ConstantTok{NA}\NormalTok{, }\ConstantTok{NA}\NormalTok{), }\AttributeTok{lwd =} \FloatTok{1.5}\NormalTok{,}
     \AttributeTok{col =} \FunctionTok{c}\NormalTok{(}\StringTok{"steelblue"}\NormalTok{, }\StringTok{"red"}\NormalTok{, }\StringTok{"brown"}\NormalTok{, }\StringTok{"gray"}\NormalTok{, }\StringTok{"gray"}\NormalTok{),}
     \AttributeTok{ylab =} \StringTok{"River flow"}\NormalTok{)}
\FunctionTok{legend}\NormalTok{(}\StringTok{"bottomleft"}\NormalTok{, }\AttributeTok{legend =} \FunctionTok{c}\NormalTok{(}\StringTok{"Observed"}\NormalTok{, }\StringTok{"Smoothed"}\NormalTok{, }\StringTok{"Forecast"}\NormalTok{, }
    \StringTok{"90\% probability limit"}\NormalTok{), }\AttributeTok{bty =} \StringTok{\textquotesingle{}n\textquotesingle{}}\NormalTok{, }\AttributeTok{pch =} \FunctionTok{c}\NormalTok{(}\ConstantTok{NA}\NormalTok{, }\ConstantTok{NA}\NormalTok{, }\ConstantTok{NA}\NormalTok{, }\ConstantTok{NA}\NormalTok{, }\ConstantTok{NA}\NormalTok{), }\AttributeTok{lty =} \DecValTok{1}\NormalTok{, }\AttributeTok{lwd =} \FloatTok{1.5}\NormalTok{,}
    \AttributeTok{col =} \FunctionTok{c}\NormalTok{(}\StringTok{"steelblue"}\NormalTok{, }\StringTok{"red"}\NormalTok{, }\StringTok{"brown"}\NormalTok{, }\StringTok{"gray"}\NormalTok{, }\StringTok{"gray"}\NormalTok{))}
\end{Highlighting}
\end{Shaded}

\pandocbounded{\includegraphics[keepaspectratio]{dlm_package_files/figure-pdf/unnamed-chunk-9-1.pdf}}

\subsection{Bonus: Implementing the Kalman filter from
scratch}\label{bonus-implementing-the-kalman-filter-from-scratch}

For the curious, the code below implements the Kalman filter from
scratch in R. Let us first implement a function
\texttt{kalmanfilter\_update} that does the update for a single time
step:

\begin{Shaded}
\begin{Highlighting}[]
\NormalTok{kalmanfilter\_update }\OtherTok{\textless{}{-}} \ControlFlowTok{function}\NormalTok{(mu, Omega, y, G, C, V, W) \{}
  
  \CommentTok{\# Prediction step {-} moving state forward without new measurement}
\NormalTok{  muPred }\OtherTok{\textless{}{-}}\NormalTok{ G }\SpecialCharTok{\%*\%}\NormalTok{ mu}
\NormalTok{  omegaPred }\OtherTok{\textless{}{-}}\NormalTok{ G }\SpecialCharTok{\%*\%}\NormalTok{ Omega }\SpecialCharTok{\%*\%} \FunctionTok{t}\NormalTok{(G) }\SpecialCharTok{+}\NormalTok{ W}
  
  \CommentTok{\# Measurement update {-} updating the N(muPred, omegaPred) prior with the new data point}
\NormalTok{  K }\OtherTok{\textless{}{-}}\NormalTok{ omegaPred }\SpecialCharTok{\%*\%} \FunctionTok{t}\NormalTok{(F) }\SpecialCharTok{/}\NormalTok{ (F }\SpecialCharTok{\%*\%}\NormalTok{ omegaPred }\SpecialCharTok{\%*\%} \FunctionTok{t}\NormalTok{(F) }\SpecialCharTok{+}\NormalTok{ V) }\CommentTok{\# Kalman Gain}
\NormalTok{  mu }\OtherTok{\textless{}{-}}\NormalTok{ muPred }\SpecialCharTok{+}\NormalTok{ K }\SpecialCharTok{\%*\%}\NormalTok{ (y }\SpecialCharTok{{-}}\NormalTok{ F }\SpecialCharTok{\%*\%}\NormalTok{ muPred)}
\NormalTok{  Omega }\OtherTok{\textless{}{-}}\NormalTok{ (}\FunctionTok{diag}\NormalTok{(}\FunctionTok{length}\NormalTok{(mu)) }\SpecialCharTok{{-}}\NormalTok{ K }\SpecialCharTok{\%*\%}\NormalTok{ F) }\SpecialCharTok{\%*\%}\NormalTok{ omegaPred}
  
  \FunctionTok{return}\NormalTok{(}\FunctionTok{list}\NormalTok{(mu, Omega))}
\NormalTok{\}}
\end{Highlighting}
\end{Shaded}

Then we implement a function that does all the Kalman iterations, using
the \texttt{kalmanfilter\_update} function above:

\begin{Shaded}
\begin{Highlighting}[]
\NormalTok{kalmanfilter }\OtherTok{\textless{}{-}} \ControlFlowTok{function}\NormalTok{(Y, G, F, V, W, mu0, Sigma0) \{}
\NormalTok{  T }\OtherTok{\textless{}{-}} \FunctionTok{dim}\NormalTok{(Y)[}\DecValTok{1}\NormalTok{]  }\CommentTok{\# Number of time steps}
\NormalTok{  n }\OtherTok{\textless{}{-}} \FunctionTok{length}\NormalTok{(mu0)  }\CommentTok{\# Dimension of the state vector}
  
  \CommentTok{\# Storage for the mean and covariance state vector trajectory over time}
\NormalTok{  mu\_filter }\OtherTok{\textless{}{-}} \FunctionTok{matrix}\NormalTok{(}\DecValTok{0}\NormalTok{, }\AttributeTok{nrow =}\NormalTok{ T, }\AttributeTok{ncol =}\NormalTok{ n)}
\NormalTok{  Sigma\_filter }\OtherTok{\textless{}{-}} \FunctionTok{array}\NormalTok{(}\DecValTok{0}\NormalTok{, }\AttributeTok{dim =} \FunctionTok{c}\NormalTok{(n, n, T))}
  
  \CommentTok{\# The Kalman iterations}
\NormalTok{  mu }\OtherTok{\textless{}{-}}\NormalTok{ mu0}
\NormalTok{  Sigma }\OtherTok{\textless{}{-}}\NormalTok{ Sigma0}
  \ControlFlowTok{for}\NormalTok{ (t }\ControlFlowTok{in} \DecValTok{1}\SpecialCharTok{:}\NormalTok{T) \{}
\NormalTok{    result }\OtherTok{\textless{}{-}} \FunctionTok{kalmanfilter\_update}\NormalTok{(mu, Sigma, }\FunctionTok{t}\NormalTok{(Y[t, ]), G, F, V, W)}
\NormalTok{    mu }\OtherTok{\textless{}{-}}\NormalTok{ result[[}\DecValTok{1}\NormalTok{]]}
\NormalTok{    Sigma }\OtherTok{\textless{}{-}}\NormalTok{ result[[}\DecValTok{2}\NormalTok{]]}
\NormalTok{    mu\_filter[t, ] }\OtherTok{\textless{}{-}}\NormalTok{ mu}
\NormalTok{    Sigma\_filter[,,t] }\OtherTok{\textless{}{-}}\NormalTok{ Sigma}
\NormalTok{  \}}
  
  \FunctionTok{return}\NormalTok{(}\FunctionTok{list}\NormalTok{(mu\_filter, Sigma\_filter))}
\NormalTok{\}}
\end{Highlighting}
\end{Shaded}

Let's try it out on the Nile river data:

\begin{Shaded}
\begin{Highlighting}[]
\CommentTok{\# Analyzing the Nile river data}
\NormalTok{prettycolors }\OtherTok{=} \FunctionTok{c}\NormalTok{(}\StringTok{"\#6C8EBF"}\NormalTok{, }\StringTok{"\#c0a34d"}\NormalTok{, }\StringTok{"\#780000"}\NormalTok{)}
\NormalTok{y }\OtherTok{=} \FunctionTok{as.vector}\NormalTok{(Nile)}
\NormalTok{V }\OtherTok{=} \DecValTok{100}\SpecialCharTok{\^{}}\DecValTok{2}
\NormalTok{W }\OtherTok{=} \DecValTok{100}\SpecialCharTok{\^{}}\DecValTok{2}
\NormalTok{mu0 }\OtherTok{=} \DecValTok{1000}
\NormalTok{Sigma0 }\OtherTok{=} \DecValTok{1000}\SpecialCharTok{\^{}}\DecValTok{2}

\CommentTok{\# Set up state{-}space model for local level model}
\NormalTok{T }\OtherTok{=} \FunctionTok{length}\NormalTok{(y)}
\NormalTok{G }\OtherTok{=} \DecValTok{1}
\NormalTok{F }\OtherTok{=} \DecValTok{1}
\NormalTok{Y }\OtherTok{=} \FunctionTok{matrix}\NormalTok{(}\DecValTok{0}\NormalTok{,T,}\DecValTok{1}\NormalTok{)}
\NormalTok{Y[,}\DecValTok{1}\NormalTok{] }\OtherTok{=}\NormalTok{ y}
\NormalTok{filterRes }\OtherTok{=} \FunctionTok{kalmanfilter}\NormalTok{(Y, G, F, V, W, mu0, Sigma0)}
\NormalTok{meanFilter }\OtherTok{=}\NormalTok{ filterRes[[}\DecValTok{1}\NormalTok{]]}
\NormalTok{std\_filter }\OtherTok{=} \FunctionTok{sqrt}\NormalTok{(filterRes[[}\DecValTok{2}\NormalTok{]][,,, }\AttributeTok{drop =}\ConstantTok{TRUE}\NormalTok{])}

\FunctionTok{plot}\NormalTok{(}\FunctionTok{seq}\NormalTok{(}\DecValTok{1}\SpecialCharTok{:}\NormalTok{T), y, }\AttributeTok{type =} \StringTok{"l"}\NormalTok{, }\AttributeTok{col =}\NormalTok{ prettycolors[}\DecValTok{1}\NormalTok{], }\AttributeTok{lwd =} \FloatTok{1.5}\NormalTok{, }\AttributeTok{xlab =} \StringTok{"time, t"}\NormalTok{)}
\FunctionTok{polygon}\NormalTok{(}\FunctionTok{c}\NormalTok{(}\FunctionTok{seq}\NormalTok{(}\DecValTok{1}\SpecialCharTok{:}\NormalTok{T), }\FunctionTok{rev}\NormalTok{(}\FunctionTok{seq}\NormalTok{(}\DecValTok{1}\SpecialCharTok{:}\NormalTok{T))), }
        \FunctionTok{c}\NormalTok{(meanFilter }\SpecialCharTok{{-}} \FloatTok{1.96}\SpecialCharTok{*}\NormalTok{std\_filter, }\FunctionTok{rev}\NormalTok{(meanFilter }\SpecialCharTok{+} \FloatTok{1.96}\SpecialCharTok{*}\NormalTok{std\_filter)), }
        \AttributeTok{col =} \StringTok{"\#F0F0F0"}\NormalTok{, }\AttributeTok{border =} \ConstantTok{NA}\NormalTok{)}
\FunctionTok{lines}\NormalTok{(}\FunctionTok{seq}\NormalTok{(}\DecValTok{1}\SpecialCharTok{:}\NormalTok{T), y, }\AttributeTok{type =} \StringTok{"l"}\NormalTok{, }\AttributeTok{col =}\NormalTok{ prettycolors[}\DecValTok{1}\NormalTok{], }\AttributeTok{lwd =} \FloatTok{1.5}\NormalTok{, }\AttributeTok{xlab =} \StringTok{"time, t"}\NormalTok{)}
\FunctionTok{lines}\NormalTok{(}\FunctionTok{seq}\NormalTok{(}\DecValTok{1}\SpecialCharTok{:}\NormalTok{T), meanFilter, }\AttributeTok{type =} \StringTok{"l"}\NormalTok{, }\AttributeTok{col =}\NormalTok{ prettycolors[}\DecValTok{3}\NormalTok{], }\AttributeTok{lwd =} \FloatTok{1.5}\NormalTok{)}
\FunctionTok{legend}\NormalTok{(}\StringTok{"topright"}\NormalTok{, }\AttributeTok{legend =} \FunctionTok{c}\NormalTok{(}\StringTok{"time series"}\NormalTok{, }\StringTok{"filter mean"}\NormalTok{, }\StringTok{"95\% intervals"}\NormalTok{), }\AttributeTok{lty =} \DecValTok{1}\NormalTok{, }\AttributeTok{lwd =} \FloatTok{1.5}\NormalTok{,}
    \AttributeTok{col =} \FunctionTok{c}\NormalTok{(prettycolors[}\DecValTok{1}\NormalTok{], prettycolors[}\DecValTok{3}\NormalTok{], }\StringTok{"\#F0F0F0"}\NormalTok{))}
\end{Highlighting}
\end{Shaded}

\pandocbounded{\includegraphics[keepaspectratio]{dlm_package_files/figure-pdf/unnamed-chunk-12-1.pdf}}




\end{document}
